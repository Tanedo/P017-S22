\documentclass[12pt]{article}
%% arXiv paper template by Flip Tanedo
%% last updated: Dec 2016



%%%%%%%%%%%%%%%%%%%%%%%%%%%%%
%%%  THE USUAL PACKAGES  %%%%
%%%%%%%%%%%%%%%%%%%%%%%%%%%%%

\usepackage{amsmath}
\usepackage{amssymb}
\usepackage{amsfonts}
\usepackage{graphicx}
\usepackage{xcolor}
\usepackage{nopageno}
\usepackage{enumerate}
\usepackage{parskip}
\usepackage{framed}
\usepackage{bbm}

\usepackage{sectsty}
\sectionfont{\Large}
% \subsectionfont{\large}
% \renewcommand{\thesection}{}
% \renewcommand{\thesubsection}{\arabic{subsection}}

%%%%%%%%%%%%%%%%%%%%%%%%%%%%%%%%%%%%%%%%%%%%%%%
%%%  PAGE FORMATTING and (RE)NEW COMMANDS  %%%%
%%%%%%%%%%%%%%%%%%%%%%%%%%%%%%%%%%%%%%%%%%%%%%%

\usepackage[margin=2cm]{geometry}   % reasonable margins

\graphicspath{{figures/}}	        % set directory for figures

% for capitalized things
\newcommand\acro[1]{{\small {#1}}}

\numberwithin{equation}{section}    % set equation numbering
\renewcommand{\tilde}{\widetilde}   % tilde over characters
\renewcommand{\vec}[1]{\mathbf{#1}} % vectors are boldface

\newcommand{\dbar}{d\mkern-6mu\mathchar'26}    % for d/2pi
\newcommand{\ket}[1]{\left|#1\right\rangle}    % <#1|
\newcommand{\bra}[1]{\left\langle#1\right|}    % |#1>
\newcommand{\Xmark}{\text{\sffamily X}}        % cross out

\let\olditemize\itemize
\renewcommand{\itemize}{
  \olditemize
  \setlength{\itemsep}{1pt}
  \setlength{\parskip}{0pt}
  \setlength{\parsep}{0pt}
}


% Commands for temporary comments
\newcommand{\comment}[2]{\textcolor{red}{[\textbf{#1} #2]}}
\newcommand{\flip}[1]{{\color{red} [\textbf{Flip}: {#1}]}}
\newcommand{\email}[1]{\texttt{\href{mailto:#1}{#1}}}

\newenvironment{institutions}[1][2em]{\begin{list}{}{\setlength\leftmargin{#1}\setlength\rightmargin{#1}}\item[]}{\end{list}}


\usepackage{fancyhdr}		% to put preprint number



%%%%%%%%%%%%%%%%%%%
%%%  HYPERREF  %%%%
%%%%%%%%%%%%%%%%%%%

%% This package has to be at the end; can lead to conflicts
\usepackage{microtype}
\usepackage[
	colorlinks=true,
	citecolor=black,
	linkcolor=black,
	urlcolor=green!50!black,
	hypertexnames=false]{hyperref}





\begin{document}


\begin{center}

    {\Large \textsc{Short HW 10}:
    \textbf{No cloning theorem}}
    
\end{center}

\vskip .4cm

\noindent
\begin{tabular*}{\textwidth}{rl}
	\textsc{Course:}& Physics 017, \emph{Linear Algebra for Physics} (S22)
	\\
	\textsc{Instructor:}& Prof. Flip Tanedo (\email{flip.tanedo@ucr.edu})
	\\
	\textsc{Due by:}& \textbf{Thursday}, June 2
\end{tabular*}

\noindent
Note that this short assignment is due by class on Thursday. You have only \emph{two days} to do it. This should be quick, I recommend doing it right after class on Tuesday.

This problem draws from the lovely text by Mertens and Moore, \emph{The Nature of Computation} (chapter 15.3). We use the basis where $|\uparrow\rangle = |0\rangle$ and $|\downarrow\rangle = |1\rangle$.

\section{No Cloning}

It turns out that it is not possible to simply copy a qubit. This poses a set of challenges for doing classical computing on a quantum computer. 

Classical computing is built on logic gates. One of the logic gates is the controlled \acro{NOT} gate, or \acro{CNOT}. This gate takes in two bits of data, $|a\rangle \otimes |b\rangle$ and spits out the following:
\begin{align}
	\text{CNOT}|a\rangle \otimes |b\rangle = 
	\begin{cases}		
		|a\rangle \otimes |b\rangle & \text{if }|a\rangle = |0\rangle\\
		|a\rangle \otimes |0\rangle & \text{if }|a\rangle\otimes|b\rangle = |1\rangle\otimes|1\rangle\\
		|a\rangle \otimes |1\rangle & \text{if }|a\rangle\otimes|b\rangle = |1\rangle\otimes |0\rangle
		\end{cases}			
\end{align}
In other words, $\text{CNOT}|a\rangle \otimes |b\rangle$ will flip the value of $|b\rangle$ if $|a\rangle = |1\rangle$. 

\subsection{CNOT as a quantum cloning gate}

Suppose you have a qubit $|\psi\rangle$ that is in some $S_z$ eigenstate: it is either $|0\rangle$ or $|1\rangle$. If you happen to have a $|0\rangle$ qubit, then show that you can \emph{clone} $\psi$ using the \acro{CNOT} gate:
\begin{align}
	\text{CNOT}|\psi\rangle\otimes |0\rangle = |\psi\rangle\otimes|\psi\rangle \ .
\end{align}
\textsc{Hint:} simply try the two cases for $|\psi\rangle$ and check what comes out.

\subsection{CNOT cannot clone superpositions}

Let us instead suppose that $|\psi\rangle$ is a superposition of $S_z$ eigenstates, e.g. 
\begin{align}
	|\psi\rangle = \psi_0|0\rangle + \psi_1|1\rangle \ .
\end{align}
Calculate $|\psi\rangle\otimes |\psi\rangle$ and argue that there is no unitary \emph{linear} operator that can possibly give this. We then conclude that $\text{CNOT}|\psi\rangle\otimes |0\rangle$ does not clone superpositions of $S_z$ eigenstates.

\textsc{Partial answer}: the proposed cloned state is
\begin{align}
 	|\psi\rangle\otimes |\psi\rangle = 
 	\psi_0^2  |0\rangle\otimes|0\rangle
	+ \psi_0 \psi_1 |0\rangle\otimes|1\rangle
	+ \psi_1 \psi_0  |1\rangle\otimes|0\rangle
	+ \psi_1 \psi_1  |1\rangle\otimes|1\rangle  \ .
 \end{align}
 The right-hand side is written with respect to an eigenbasis of the tensor product. Compare this to the initial state, $|\psi\rangle\otimes|0\rangle$. Is there any linear operator that can take something linear in $\psi_1$ and $\psi_2$ and then turn it into something that is not linear in $\psi_1$ and $\psi_2$?







\end{document}
