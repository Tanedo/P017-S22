\documentclass[12pt]{article}
%% arXiv paper template by Flip Tanedo
%% last updated: Dec 2016



%%%%%%%%%%%%%%%%%%%%%%%%%%%%%
%%%  THE USUAL PACKAGES  %%%%
%%%%%%%%%%%%%%%%%%%%%%%%%%%%%

\usepackage{amsmath}
\usepackage{amssymb}
\usepackage{amsfonts}
\usepackage{graphicx}
\usepackage{xcolor}
\usepackage{nopageno}
\usepackage{enumerate}
\usepackage{parskip}
\usepackage{framed}
\usepackage{bbm}

\usepackage{sectsty}
\sectionfont{\Large}
% \subsectionfont{\large}
% \renewcommand{\thesection}{}
% \renewcommand{\thesubsection}{\arabic{subsection}}

%%%%%%%%%%%%%%%%%%%%%%%%%%%%%%%%%%%%%%%%%%%%%%%
%%%  PAGE FORMATTING and (RE)NEW COMMANDS  %%%%
%%%%%%%%%%%%%%%%%%%%%%%%%%%%%%%%%%%%%%%%%%%%%%%

\usepackage[margin=2cm]{geometry}   % reasonable margins

\graphicspath{{figures/}}	        % set directory for figures

% for capitalized things
\newcommand\acro[1]{{\small {#1}}}

\numberwithin{equation}{section}    % set equation numbering
\renewcommand{\tilde}{\widetilde}   % tilde over characters
\renewcommand{\vec}[1]{\mathbf{#1}} % vectors are boldface

\newcommand{\dbar}{d\mkern-6mu\mathchar'26}    % for d/2pi
\newcommand{\ket}[1]{\left|#1\right\rangle}    % <#1|
\newcommand{\bra}[1]{\left\langle#1\right|}    % |#1>
\newcommand{\Xmark}{\text{\sffamily X}}        % cross out

\let\olditemize\itemize
\renewcommand{\itemize}{
  \olditemize
  \setlength{\itemsep}{1pt}
  \setlength{\parskip}{0pt}
  \setlength{\parsep}{0pt}
}


% Commands for temporary comments
\newcommand{\comment}[2]{\textcolor{red}{[\textbf{#1} #2]}}
\newcommand{\flip}[1]{{\color{red} [\textbf{Flip}: {#1}]}}
\newcommand{\email}[1]{\texttt{\href{mailto:#1}{#1}}}

\newenvironment{institutions}[1][2em]{\begin{list}{}{\setlength\leftmargin{#1}\setlength\rightmargin{#1}}\item[]}{\end{list}}


\usepackage{fancyhdr}		% to put preprint number



%%%%%%%%%%%%%%%%%%%
%%%  HYPERREF  %%%%
%%%%%%%%%%%%%%%%%%%

%% This package has to be at the end; can lead to conflicts
\usepackage{microtype}
\usepackage[
	colorlinks=true,
	citecolor=black,
	linkcolor=black,
	urlcolor=green!50!black,
	hypertexnames=false]{hyperref}





\begin{document}


\begin{center}

    {\Large \textsc{Short HW 8}:
    \textbf{The exponential basis for Fourier Series}}
    
\end{center}

\vskip .4cm

\noindent
\begin{tabular*}{\textwidth}{rl}
	\textsc{Course:}& Physics 017, \emph{Linear Algebra for Physics} (S22)
	\\
	\textsc{Instructor:}& Prof. Flip Tanedo (\email{flip.tanedo@ucr.edu})
	\\
	\textsc{Due by:}& \textbf{Thursday}, May 19
\end{tabular*}

\noindent
Note that this short assignment is due by class on Thursday. You have only \emph{two days} to do it. This should be quick, I recommend doing it right after class on Tuesday.

In class we presented a basis of complex functions defined from $-L\leq x\leq L$:
\begin{align}
	|n\rangle = \sqrt{\frac{1}{2L}}e^{-\frac{in\pi}{L} x} \ .
\end{align}
A general function $f = |f\rangle$ is expanded as follows:
\begin{align}
	|f\rangle = \sum_{n=-\infty}^\infty c_n |n\rangle \ ,
\end{align}
where we note that the sum extends over all integers from minus to plus infinity. We have not yet specified boundary conditions at $x=\pm L$. This imposes relationships between the coefficients $c_n$.

\section{Reproducing the sine series}

Suppose our function space has Dirichlet boundary conditions:
\begin{align}
	f(L) &= 0 & f(-L) &= 0 \ .
\end{align}
What condition does this impose on the $c_n$? \textsc{Hint:} you should remember that these boundary conditions are satisfied by a Fourier sine series. What relations do $c_n$ and $c_{-n}$ satisfy for the expansion to be a sum of sine functions? 



\section{Orthonormality}
\subsection{Orthogonality}
Show that $\langle n|m\rangle = 0$ for $n\neq m$.

\subsection{Normality}
Show that $\langle n | n\rangle =1$. Explain why this integral is non-zero, even though the previous integral for $\langle n|m\rangle$ is zero for $n\neq m$. 

\section{But my textbook says something different...}

If you look up the complex Fourier series in some physics textbooks\footnote{e.g.\ Felder \& Felder, \emph{Mathematical Methods in Engineering and Physics},  (9.5.2).}, you may find a different expansion:
\begin{align}
	f(x) &= \sum_{n=-\infty}^\infty c_n e^{-\frac{in\pi}{L} x}
	&
	c_n &= \frac{1}{2L} \int_{-L}^L f(x)^* \, e^{-\frac{in\pi}{L} x} \ .
\end{align}
This looks like the normalization is totally different from our conventions! Show that the textbook rules above give the same Fourier series as our conventions. 

\textsc{Comment:} our convention is better because our basis functions are normalized. The textbook definition uses not-normalized basis functions, and as a result the coefficients $c_n$ swallow part of the normalization factor.


\end{document}
