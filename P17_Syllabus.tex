\documentclass[12pt]{article}

\makeatletter
\renewcommand{\@seccntformat}[1]{} 	% remove sections
\makeatother

% footer with a link (argument has no `https://'')
\newcommand{\footlink}[1]{\footnote{\href{https://#1}{\texttt{#1}}}}

\newcommand{\UCR}{\acro{UCR}\xspace}
\newcommand{\UCRiverside}{\acro{UC~R}iverside\xspace}


%%%%%%%%%%%%%%%%%%%%%%%%%%%%%
%%%  THE USUAL PACKAGES  %%%%
%%%%%%%%%%%%%%%%%%%%%%%%%%%%%

\usepackage{amsmath}
\usepackage{amssymb}
\usepackage{amsfonts}
\usepackage{graphicx}
\usepackage{xcolor}
\usepackage{xspace}
\usepackage{nopageno}


%%%%%%%%%%%%%%%%%%%%%%%%%%%%%%%%%%%%%%%%%%%%%%%
%%%  PAGE FORMATTING and (RE)NEW COMMANDS  %%%%
%%%%%%%%%%%%%%%%%%%%%%%%%%%%%%%%%%%%%%%%%%%%%%%

\usepackage[margin=2cm]{geometry}
\graphicspath{{figures/}}

% for capitalized things
% \newcommand{\acro}[1]{\textsc{\MakeLowercase{#1}}}

% \usepackage{scalefnt} 
% \newcommand\acro[2][0.875]{{\scalefont{#1}#2}} 
\newcommand\acro[1]{{\small {#1}}}

\numberwithin{equation}{section}    % set equation numbering
\renewcommand{\tilde}{\widetilde}   % tilde over characters
\renewcommand{\vec}[1]{\mathbf{#1}} % vectors are boldface

\newcommand{\dbar}{d\mkern-6mu\mathchar'26}    % for d/2pi
\newcommand{\ket}[1]{\left|#1\right\rangle}    % <#1|
\newcommand{\bra}[1]{\left\langle#1\right|}    % |#1>
\newcommand{\Xmark}{\text{\sffamily X}}        % cross out

% Change list spacing (instead of package paralist)
\let\oldenumerate\enumerate
\renewcommand{\enumerate}{
  \oldenumerate
  \setlength{\itemsep}{1pt}
  \setlength{\parskip}{0pt}
  \setlength{\parsep}{0pt}
}

\let\olditemize\itemize
\renewcommand{\itemize}{
  \olditemize
  \setlength{\itemsep}{1pt}
  \setlength{\parskip}{0pt}
  \setlength{\parsep}{0pt}
}


%%%%%%%%%%%%%%%%%%%
%%%  HYPERREF  %%%%
%%%%%%%%%%%%%%%%%%%

%% This package has to be at the end; can lead to conflicts
\usepackage{microtype}
\usepackage[
	colorlinks=true,
	citecolor=black,
	linkcolor=black,
	urlcolor=green!50!black,
	hypertexnames=false]{hyperref}



%%%%%%%%%%%%%%%%%%%%%
%%%  TITLE DATA  %%%%
%%%%%%%%%%%%%%%%%%%%%

\begin{document}
\begin{center}

    {\Large \textsc{Physics 017:} \textbf{Linear Algebra for Physics}, Spring 2022}

\end{center}

% \section{Course Data}

\noindent
\begin{tabular*}{\textwidth}{rlcrl}
\textsc{Lec:}& Prof.~Flip Tanedo (\texttt{flip.tanedo@ucr.edu})
&
\hspace{.5cm}
&
\textsc{Meet:} & 
TR 12:30--1:50pm \textsc{phys} 2104
\\
\textsc{ta:}& Ian Chaffey (\texttt{ichaf001@ucr.edu})
&
\hspace{.25cm}
&
\textsc{Dis:} & none
\\
\textsc{crn:} & 69513 (Course Reference Number)
&
\hspace{.25cm}
&
\textsc{units:} & 4
\end{tabular*}


% Associated Term: Spring 2022
% CRN: 69513
% Campus: Riverside
% Schedule Type: Lecture
% Instructional Method: In-Person
% Section Number: 001
% Subject: Physics
% Course Number: 017
% Title: LINEAR ALGEBRA FOR PHYSICS
% Units: 4

% 4 Units, Lecture, 3 hours; discussion, 1 hour. Prerequisite(s): MATH 010B with a grade of C- or better, MATH 046 with a grade of C- or better. This course covers the essential mathematics for quantum mechanics at the upper-division level. It applies linear algebra to finite and infinite dimensional vector spaces. Topics include: matrices, linear equations, bases, eigenvectors and eigenvalues, functions as infinite-dimensional vectors, differential operators as matrices, Fourier transforms, and eigenfunctions of common differential operators.

\subsection*{Critical Information}
\textsc{web page:} \texttt{\href{https://sites.google.com/ucr.edu/physics017}{sites.google.com/ucr.edu/physics017}}\\
\textsc{canvas:} \texttt{\href{https://elearn.ucr.edu/courses/37232}{elearn.ucr.edu/courses/37232}}


\vspace{.5em}
\noindent Lecture notes, homework and our course calendar will be linked from the course web page. We will post some course readings (e.g. selections from textbooks) on the Canvas page.
%
The professor and \acro{TA} office hours will be posted on the course webpage. 

\vspace{.5em}
\noindent This document was last updated: \today


\section{Course Materials}

Unfortunately there is no perfect textbook for this course.\footnote{On the other hand, this means we have the freedom to teach the course according to how physicists actually think about linear algebra, tensors, and all that.} Rather than forcing you to purchase an expensive book that does not align with our focus, we will post lecture notes online and supplement them with selections from other textbooks on Canvas.
\vspace{.5em}

% \subsection{Required and Suggested Textbooks}

\noindent\textbf{Primary}: Course notes are posted to our website and will be our primary reference. We also post selected readings on our course Canvas page as additional references. 
\vspace{.5em}

\noindent\textbf{Suggested}: The first chapter (``Mathematical Introduction'') of \emph{Principles of Quantum Mechanics} (2nd ed.)  by R.~Shankar covers all of the main ideas in our course. Selections from this chapter posted to Canvas.
\vspace{.5em}

\noindent\textbf{Suggested}: The appendix of \emph{Introduction to Quantum Mechanics} (2nd ed.)  by D.~Griffiths has a similar introduction to these key ideas at a slightly more accessible level. Selections from this chapter posted to Canvas.
\vspace{.5em}

\noindent\textbf{Suggested}: Course notes, ``\emph{The Mathematics of Quantum Mechanics}'' by M.~LaForest covers a broad set of ideas in a similar style. Selections from this chapter posted to Canvas.
\vspace{.5em}

\noindent\textbf{Optional}: \emph{Linear Algebra Done Right}, Axler. [Digital version available free through UCR library.] This is a more formal linear algebra textbook from a more mathematical perspective. If you are especially mathematical you may want to follow along with this book.
\vspace{.5em}

% \noindent Additional reading: chapters from \emph{Mathematics for Quantum Mechanics: An Introductory Survey of Operators, Eigenvalues, and Linear Vector Spaces}, Jackson.



\section{Course Description}

Physics~17 covers the essential mathematics for quantum mechanics for undergraduate physics majors. It introduces real and complex linear algebra of finite and infinite-dimensional vector spaces. Examples are drawn from simple quantum mechanical systems. \textbf{Topics may include}: matrices, linear equations, bases, eigenvectors and eigenvaues, complex vector spaces, functions as infinite-dimensional vectors, differential operators as matrices, Fourier transforms, eigenfunctions of common differential operators, and basics of the representation theory of Lie groups.

% and \subsection{Comparison to Other Courses}
\subsection{Course Prerequisites}

There are no strict prerequisites. We expect students to have taken the equivalent of:
\begin{itemize}
	\item single variable calculus (Math 9)
	\item the first-year physics sequence (Physics 40 or 41)\ .
\end{itemize}
You should have an idea of matrix multiplication from precalculus. Note: the current course catalog lists Math ~46 and 10B as prerequisites for this course. The instructor is willing to waive these requirements for students on a case-by-case basis.

\subsection{Course Format}

This course will be \emph{in person}. 

\subsection{Course Expectations}

I \emph{strongly} encourage you to ask questions in class and to discuss with your classmates outside of class. There are two questions that you can \emph{always} ask:
\begin{enumerate}
	\item ``\emph{Is it obvious that...?}'' This means: I don't know if I fully understand something. Maybe I'm looking at it the wrong way, what is the best way to see that this is true? 
	\item ``\emph{Why are we doing this?}'' You may understand the details, but have lost track of the big picture. What is the main point of this section?
\end{enumerate}
These are good ways to clarify what we're doing without worrying about ``appearing dumb'' for asking them.

\subsection{Course Policies}


\begin{itemize}
	\item \textbf{Inclusion}: we are committed to creating an inclusive learning space where we respect one another regardless of race/ethnicity, gender identities, gender expressions, sexual orientation, socioeconomic status, age, disabilities, religion, regional background, veteran status, citizenship status, nationality and other diverse identities that we each bring to class.
	\item \textbf{No bullying}: this course requires students to share work with one another. We will treat each other with respect in our constructive criticism and we will not share each others' materials outside our course without their explicit written permission. Do not be a troll or a bully; we are each offering some vulnerability to support this learning environment. The instructors reserve the right to punish misbehavior with zero credit on assignments or failure in the course. Be kind.
	\item \textbf{Attendance}: attendance is not part of your grade but is strongly encouraged. You are investing time and resources to be a \UCR student and take this class; the goal of the instructors is for the class time to be valuable rather than simply being a recitation of something you could learn by reading a book. If the instructors are failing at this goal, please contact them to politely inform them of this shortcoming. 
	\item \textbf{Late homework will not be accepted}. The peer critique aspect of this course is only fair if your peers have access to your work in a timely manner.  
	\item \textbf{Communication policy}: When emailing the instructors, please include \textbf{[P017]} in the subject line. For example, ``\texttt{[P017] Possible typo on the homework assignment}''. This ensures that the email will be routed properly. The professor anticipates checking course email twice a week on the day of lecture. Physics questions should be asked \emph{in class} where everyone can benefit from the discussion.
\end{itemize}

\subsection{COVID-19 Policies}

We will follow the \UCR~\acro{COVID} return protocols\footlink{campusreturn.ucr.edu}. Please note that we will not be recording lectures. If you have to miss lectures, please confer with your classmates and the posted course notes. 

\section*{Course Requirements and Assessment}

This course is 4 units. One course unit corresponds to approximately three hours of student time per week. Thus 4 units = 12 hours/week, which roughly translates into 3 hours/week of lecture time and 9 hours/week of out-of-class reading, writing, discussing, and recording.  

\paragraph{Assignments.} All assignments are submitted electronically by an online form. Please typeset or scan your work (e.g.~via phone).
\begin{itemize}
	\item \textbf{Short homework}: assigned every Tuesday, due on Thursday before the start of class. These are \emph{brief} exercises to review earlier material.
	\item \textbf{Long homework}: assigned on Tuesday of odd-numbered weeks and due in two weeks. You will have a total of four assignments.
	\item \textbf{Explainer video}:  assigned on Tuesday of odd-numbered weeks and due in two weeks. For each long homework, you will be semi-randomly assigned one problem to prepare a 5-minute video explaining the solution to your classmates. 
	\item \textbf{Peer review}: assigned on Thursday of odd-numbered weeks and due in one week. You will review some of your classmates explainer videos and provide feedback.
\end{itemize}

\paragraph{Grading.} There will be no final exam and we will not use the final exam slot.
\begin{itemize}
\item 20\% Short homework
\item 30\% Long homework
\item 30\% Explainer video
\item 20\% Peer review
\end{itemize}
Occasional extra credit opportunities may come up in class and in the assignments.

% \noindent The course discussion section will be used primarily to do example problems, answer questions about the class, address questions on the homework, and prepare for the examinations.



\section{Learning Objectives}

By the end of this course, you are expected to be able to do the the following learning outcomes:
\begin{enumerate}
	\item Write a system of linear equations as a matrix equation.
	\item Apply the Gram-Schmidt procedure to produce an orthonormal basis.
	\item Interpret tensors as multilinear maps and perform index-based tensor manipulations on metric spaces.
	\item Interpret Lorentz transformations as rotations in Minkowski space.
	\item Systematically diagonalize a matrix into a basis of eigenvectors and identify the eigenvalues.
	\item Write vectors and matrices using bra--ket notation.
 	\item Connect the eigenvectors and eigenvalues of a quantum mechanical system to physical observables and states.
 	\item Interpret differential operators as matrices on an infinite dimensional vector space.
	\item Perform Fourier transformations to momentum space to derive the eigenfunctions and eigenvalues of differential operators. 
	\item Identify whether an operator may be a physical observable based on hermiticity.
	\item Confirm the orthnormality of the eigenfunctions of a differential operator.
	\item Relate mathematical properties of operators to physical properties in quantum mechanics.
\end{enumerate}

\noindent By the end of this course, students will be mathematically prepared for upper-division quantum mechanics and electrodynamics.



% \section{Grading}




\section{Syllabus/Agenda}

The weekly agenda is as follows. 
% The `work' for the week is the focus of the examples in class and homework assignments. 
We reserve the right to adapt the topics and pacing as needed for our class. Please see the course webpage for the most updated information.

\begin{enumerate}
	\item %1 
	Matrices, index notation, basis vectors. Review of `high school' vectors and matrices. All sorts of notation for vector spaces.
	\\
	\textsf{\small 
		\textbf{Reading}: Laforest 2.1--2.5, Shankar 1.1.
		% \\
		% \textbf{Work}: 
	}

	\item %2
	Metrics: raising and lowering indices. Changing basis.
	\\
	\textsf{\small 
		\textbf{Reading}: Laforest 2.5--2.6, Shankar 1.2.
		% \\
		% \textbf{Work}: 
	}

	\item %3
	Linear algebra and (multi-)linear maps.
	\\
	\textsf{\small 
		\textbf{Reading}: Axler Ch.~3, Shankar 1.3.
		% \\
		% \textbf{Work}: 
	}

	\item %4
	Rotations and boosts. A hint of complex linear algebra. Special relativity.
	\\
	\textsf{\small 
		\textbf{Reading}: to be determined, e.g.~Ch.~1 from \emph{Spacetime and Geometry} by S.~Carroll. Shankar 1.4--1.7
		% \\
		% \textbf{Work}: 
	}

	\item %5
	Eigenvalues and Eigenvectors.
	\\
	\textsf{\small 
		\textbf{Reading}: Shankar 1.8. Griffiths A.5.
		% \\
		% \textbf{Work}: 
	}

	\item %6
	Complex vector spaces. Spinors, a hint of quantum mechanics.
	\\
	\textsf{\small 
		\textbf{Reading}: Shankar 1.4--1.7
		% \\
		% \textbf{Work}: 
	}

	\item %7
	Eigenvalue/eigenvector problems in quantum mechanics. Commutators.
	\\
	\textsf{\small 
		\textbf{Reading}: to be determined, e.g.~Griffiths appendix.
		% \\
		% \textbf{Work}: 
	}

	\item %8
	Infinite dimensional vector spaces. Function spaces, differential operators. Differential equations as linear algebra with very large matrices. Fourier transforms.
	\\
	\textsf{\small 
		\textbf{Reading}: to be determined, e.g.~Shankar 1.9.
		% \\
		% \textbf{Work}: 
	} 

	\item %9
	Representation theory of Lie groups. Clebsch--Gordan coefficients. Tensor products.
	\\
	\textsf{\small 
		\textbf{Reading}: to be determined.
		% \\
		% \textbf{Work}: 
	} 

	\item Room for special topics according to interest. Possible topics include the basics of quantum computing and quantum information, Green's functions in field theory, a hint of general relativity based on symmetry principles.
\end{enumerate}






\section{UC Riverside Policies}

The information below is fairly standard and you may have already read much of it several times in different courses.\footnote{Much of the following text is adapted from templates from the \acro{UCR XCITE} team. Yes, this is a citation.} Please be aware of topics where this course differs from others, for example, academic integrity policies with respect to solutions sites. 

\subsection{Course-specific norms}

\paragraph{Titles.} We ask that you refer to the professor as \emph{Dr.~Tanedo} or \emph{Professor Tanedo}. The purpose of this norm is to encourage uniform honorifics and avoid gender-based and ethnicity-based stereotyping that can affect early career academics.

\subsection{Academic Misconduct and Plagiarism}

It is the \emph{responsibility of each student} to be familiar with the definitions, policies, and procedures concerning academic misconduct. Please revisit our Academic Integrity Policies and Procedures\footlink{conduct.ucr.edu/policies/academic-integrity-policies-and-procedures} for more information. This site also defines misconduct, provides examples of prohibited conduct, and explains the sanctions available for those found guilty of misconduct.

Plagiarism is the appropriation of another person's ideas, processes, results, or words without giving appropriate credit. This includes the copying of language, structure, or ideas of another and attributing (explicitly or implicitly) the work to one's own efforts. Issues of significant academic integrity will be reported to the Student Conduct \& Academic Integrity Programs office\footlink{conduct.ucr.edu} for mediation. The \acro{SCAIP} office adjudicates these cases independently of the instructor.

A rough guideline for \emph{this course} is:
\begin{itemize}
	\item You are \emph{encouraged} to work with other students. You should ask each other questions and learn from one another. Your submitted work should draw on your collaborations, but should be prepared by yourself independently.
	\item You are \emph{encouraged} to use reputable outside resources. Anything that is not explicitly part of our course materials should be \emph{cited}. A citation contains enough information for any reader to access the same resource. Even after citing other work, your submitted work should be prepared by yourself independently.
	\item You are \emph{welcome} to have others give you feedback on your independently prepared work and to incorporate that feedback into revisions of your work. All revisions should reflect your own independent work based on that feedback.
	\item \emph{What about Chegg (and similar sites)?} The course materials contain several worked examples that are the best practice problems for our class. Given that we encourage outside research, we will \emph{not} restrict what resources you may use---however, we do discourage lazy use of solutions sites like Chegg. If you do use any external resources---whether Chegg, Wikipedia, other textbooks, or an academic journal---you are \emph{required} to cite those sources in your submitted assignment. 
\end{itemize}
For example: while researching a particular problem, you may find a part of the problem solved in a clever way on some solutions site. You are allowed to learn from this and think carefully for how it applies to the specific problem on your assignment. You should prepare your assignment \emph{without} the site in front of you, and in your solution you should provide a a citation to the webpage along the lines of: ``\emph{the technique to use a Fourier transform to integrate this distribution comes from Person McPersonson via \texttt{https://...}}'' You are responsible for the intellectual merit of anything you submit---so if your solution draws from a garbage source, you can expect a garbage grade.

\paragraph{Motivation for this academic integrity policy.} Part of your training as an academic is to effectively use existing literature to solve new problems. 100 years ago `existing literature' meant printed material in a library. In modern times this includes everything accessible on the web. Rather than trying to artificially regulate permissible sources, we will strive for meaningful assessment and responsible engagement with the breadth of possible sources. Proper citation is not only honesty about sources of inspiration; it also highlights your individual contributions to your finished product. 
%
Ultimately, your submitted assignments are for your benefit. Careful citations to outside literature will make it easy for you to refer back to useful resources in the future. 

\paragraph{Summary:} When in doubt about academic integrity in this course, you may contact the professor.


\subsection{Academic Support Services}


\paragraph{Undergraduates:} \UCR Academic Resource Center\footlink{arc.ucr.edu} offers peer-to-peer as well as staff supported mentoring and workshops. You may find it helpful to access their writing center\footlink{arc.ucr.edu/writing}. 

\paragraph{Graduates:} refer to the \UCR Graduate Student Resource Center\footlink{gsrc.ucr.edu} and Graduate Writing Center\footlink{gwc.ucr.edu}.

\paragraph{Student Needs:} The following \UCR resources are available to support students:
\begin{itemize}
	\item Student Health Services\footlink{studenthealth.ucr.edu}
	\item Counseling \& Psychological Services\footlink{counseling.ucr.edu} (\acro{CAPS})
	\item Residential Life and Dining: Basic Needs\footlink{basicneeds.ucr.edu}, including R’Pantry and Emergency Housing
\end{itemize}



\subsection{Equity and Access}

\subsubsection{Reasonable Accommodation for Disabilities}
Students with disabilities who require accommodations in this course must be registered with Student Disability Resource Center (\acro{SDRC}).\footlink{sdrc.ucr.edu} The \acro{SDRC} directly contacts the professor to handle all disability-related accommodations.
%
If you have a disability and you would like to make a request for reasonable accommodation, please contact the Student Disability Resource Center directly.


\subsubsection{Adjustments for Pregnancy/Childbirth}

Should you need modifications or adjustments to your course requirements because of documented pregnancy-related or childbirth-related issues, please contact the instructor. Generally, modifications will be made where medically necessary and similar in scope to accommodations based on temporary disability.  Learn more about the rights of pregnant and parenting students by consulting the Office of Diversity, Equity, and Inclusion.\footlink{diversity.ucr.edu}

\subsubsection{Title IX Resources}

For any concerns regarding gender-based discrimination, sexual harassment, sexual misconduct, stalking, or intimate partner violence, the University offers a variety of resources, including advocates on-call 24/7, counseling services, mutual no contact orders, scheduling adjustments, and disciplinary sanctions against the perpetrator. Please see the Title IX website\footlink{titleix.ucr.edu} for more information or to file a report. They can be reached at (951)827-7070. 

Please note that faculty and staff at \UCR who receive information regarding sexual violence and sexual harassment are required to share that information with the Title IX office, for example by making their own report. If you are need a confidential source of support, please reach out to the \acro{ucr care} office.\footlink{care.ucr.edu}

\subsubsection{Religious Absences}

It is the policy of the University to excuse absences of students that result from religious observances and to provide for the rescheduling of examinations and additional required classwork that may fall on religious holidays without penalty. It is the responsibility of \emph{the student} to make alternate arrangements with the instructor at least one week prior to the actual date of the religious holiday.



\subsection{Library And Technical Support}

Access digital materials and other resources at the \UCR Library\footlink{library.ucr.edu}. It may be especially helpful to be able to connect to the campus network through the \UCR~\acro{VPN}\footlink{library.ucr.edu/using-the-library/technology-equipment/connect-from-off-campus}.

\acro{ITS} Student Technology Services\footlink{its.ucr.edu/sts} supports 9 student computer labs, including 7 public labs and 2 nonpublic labs, with approximately 293 public lab hours per week available for academic use by all \UCR students. \UCR students can use the \acro{ITS STS} website to submit a ServiceLink ticket or may directly contact \texttt{BearHelp@ucr.edu}. They provide assistance with computer and specialized software access, R'Mail and Canvas accounts, network connectivity, or any other services used by \UCR students.

\subsection{Recognition of reading the syllabus}

In recognition of anyone who actually skimmed through the syllabus (it was a pain to write) and made it this far, please email the professor with the following subject title: ``[P017] Capybara Parade'' and attach your favorite photo of a capybara (for good practice, please include a citation to the source). Students who do this before Week 2 will receive a small extra credit on their course grade.

\subsection{Registration and Withdrawal}

If you choose to withdraw from this course, you must complete the appropriate forms from the Office of the Registrar.\footlink{registrar.ucr.edu/calendar}


\end{document}
