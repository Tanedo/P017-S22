\documentclass[12pt]{article}
%% arXiv paper template by Flip Tanedo
%% last updated: Dec 2016



%%%%%%%%%%%%%%%%%%%%%%%%%%%%%
%%%  THE USUAL PACKAGES  %%%%
%%%%%%%%%%%%%%%%%%%%%%%%%%%%%

\usepackage{amsmath}
\usepackage{amssymb}
\usepackage{amsfonts}
\usepackage{graphicx}
\usepackage{xcolor}
\usepackage{nopageno}
\usepackage{enumerate}
\usepackage{parskip}
\usepackage{framed}

\usepackage{sectsty}
\sectionfont{\Large}
% \subsectionfont{\large}
% \renewcommand{\thesection}{}
% \renewcommand{\thesubsection}{\arabic{subsection}}

%%%%%%%%%%%%%%%%%%%%%%%%%%%%%%%%%%%%%%%%%%%%%%%
%%%  PAGE FORMATTING and (RE)NEW COMMANDS  %%%%
%%%%%%%%%%%%%%%%%%%%%%%%%%%%%%%%%%%%%%%%%%%%%%%

\usepackage[margin=2cm]{geometry}   % reasonable margins

\graphicspath{{figures/}}	        % set directory for figures

% for capitalized things
\newcommand\acro[1]{{\small {#1}}}

\numberwithin{equation}{section}    % set equation numbering
\renewcommand{\tilde}{\widetilde}   % tilde over characters
\renewcommand{\vec}[1]{\mathbf{#1}} % vectors are boldface

\newcommand{\dbar}{d\mkern-6mu\mathchar'26}    % for d/2pi
\newcommand{\ket}[1]{\left|#1\right\rangle}    % <#1|
\newcommand{\bra}[1]{\left\langle#1\right|}    % |#1>
\newcommand{\Xmark}{\text{\sffamily X}}        % cross out

\let\olditemize\itemize
\renewcommand{\itemize}{
  \olditemize
  \setlength{\itemsep}{1pt}
  \setlength{\parskip}{0pt}
  \setlength{\parsep}{0pt}
}


% Commands for temporary comments
\newcommand{\comment}[2]{\textcolor{red}{[\textbf{#1} #2]}}
\newcommand{\flip}[1]{{\color{red} [\textbf{Flip}: {#1}]}}
\newcommand{\email}[1]{\texttt{\href{mailto:#1}{#1}}}

\newenvironment{institutions}[1][2em]{\begin{list}{}{\setlength\leftmargin{#1}\setlength\rightmargin{#1}}\item[]}{\end{list}}


\usepackage{fancyhdr}		% to put preprint number



%%%%%%%%%%%%%%%%%%%
%%%  HYPERREF  %%%%
%%%%%%%%%%%%%%%%%%%

%% This package has to be at the end; can lead to conflicts
\usepackage{microtype}
\usepackage[
	colorlinks=true,
	citecolor=black,
	linkcolor=black,
	urlcolor=green!50!black,
	hypertexnames=false]{hyperref}





\begin{document}


\begin{center}

    {\Large \textsc{Short HW 2}:
    \textbf{A Change of Basis}}
    
\end{center}

\vskip .4cm

\noindent
\begin{tabular*}{\textwidth}{rl}
	\textsc{Course:}& Physics 017, \emph{Linear Algebra for Physics} (S2022)
	\\
	\textsc{Instructor:}& Prof. Flip Tanedo (\email{flip.tanedo@ucr.edu})
	\\
	\textsc{Due by:}& \textbf{Thursday}, April 7 (yes, \emph{this} Thursday)
\end{tabular*}

\noindent
Note that this short assignment is due by class on Thursday. You have only \emph{two days} to do it. This should be quick, I recommend doing it right after class on Tuesday.

The goal of this assignment is to develop some intuition for changes of basis. In this class (and in most of physics) change from one orthonormal basis\footnote{This means that the basis vectors are all unit length and perpendicular to each other. The ``unit length'' statement is kind of funny; you can define the metric to have inherited the notion of length from its basis vectors.} to another. These are \emph{rotations} and generalizations of rotations, like Lorentz transformations\footnote{The appropriate mathematical word is probably \emph{isometry}, the combination of \emph{iso}- meaning ``same'' and \emph{metry}, the same root used for metric.}.

\section{From the unprimed to primed basis}

Let $\hat{\vec{e}}_{(1)}$ and $\vec{e_{(2)}}$ be the standard basis of $\mathbb{R}^2$:
\begin{align}
	\hat{\vec{e}}_{(1)}&=
	\begin{pmatrix}
		1 \\ 0
	\end{pmatrix}
	&
	\hat{\vec{e}}_{(2)}&=
	\begin{pmatrix}
		0 \\ 1
	\end{pmatrix} \ .
\end{align}
Define a primed basis $\hat{\vec{e}}'_{(1)}$ and $\vec{e'_{(2)}}$ be the standard basis of $\mathbb{R}^2$:
\begin{align}
	\hat{\vec{e}}'_{(1)}&=
	\begin{pmatrix}
		 1 \\ 1
	\end{pmatrix}
	&
	\hat{\vec{e}}'_{(2)}&=
	\frac{1}{\sqrt{2}}
	\begin{pmatrix}
		-1 \\ \phantom{+}1
	\end{pmatrix} \ .
\end{align}

\subsection{Draw!}

Draw both pairs of vectors on the Cartesian plain. 

\subsection{Calculate}

% The Euclidean metric is $g_{ij} = \delta_{ij}$ so that:
% \begin{align}
% 	g_{11} &= g_{22} = 1 
% 	&
% 	g_{12} &= g_{21} = 0 \ .
% \end{align}

Let us define the usual Cartesian \emph{dot product} between unit-length vectors $\hat{\vec{v}}$ and $\hat{\vec{w}}$ to be 
\begin{align}
	\hat{\vec{v}}\cdot \hat{\vec{w}} = \cos\theta \ ,
\end{align}
where $\theta$ is the angle between $\hat{\vec{v}}$ and $\hat{\vec{w}}$. Please write out the following dot products:
\begin{align}
	\hat{\vec{e}}'_{(1)}\cdot \hat{\vec{e}}_{(1)}
	&&
	\hat{\vec{e}}'_{(1)}\cdot \hat{\vec{e}}_{(2)}
	&&
	\hat{\vec{e}}'_{(2)}\cdot \hat{\vec{e}}_{(1)}
	&&
	\hat{\vec{e}}'_{(2)}\cdot \hat{\vec{e}}_{(2)} \ .
	\label{eq:basisdots}
\end{align}

\subsection{A vector appears}\label{sec:avector}

I define the vector $\vec{v}$ with respect to its components in the unprimed basis $v^i$:
\begin{align}
	\vec{v}
	=
	v^1 \hat{\vec{e}}_{(1)} + v^2\hat{\vec{e}}_{(2)}
	=
	3 \hat{\vec{e}}_{(1)} + 2\hat{\vec{e}}_{(2)} \ .
 \ .
\end{align}
We usually write this as
\begin{align}
	\begin{pmatrix}
		v^1\\ v^2
	\end{pmatrix}
	=
	\begin{pmatrix}
		3\\
		2
	\end{pmatrix}
\end{align}
First, draw this vector on the Cartesian plane. On the same plane, draw the primed basis vectors, $\hat{\vec{e}}'_{(1)}$ and $\hat{\vec{e}}'_{(2)}$. 

\subsection{Writing the vector in the primed basis}

The vector $\vec{v}$ can alternatively be written with respect to its components $v'^i$ in the primed basis:
\begin{align}
	\vec{v} 
	= v'^1 \hat{\vec{e}}'_{(1)} + v'^2\hat{\vec{e}}'_{(2)}
	\ . \label{eq:primed}
\end{align}
Find the components $v'^i$. The picture in \ref{sec:avector} should help, but think about how you would do this problem if you did not have a picture.


\section*{Discussion}

\subsection*{Mathematicians giggle at us sometimes}

Pure mathematicians are sometimes baffled by physicists' obsession with components and indices. To a pure mathematician, the vector $\vec{v}$ is just that: an object with some notion of being an abstract object in an abstract \textbf{vector space}. They will begrudgingly accept that the vector space can have a basis, and so we may write
\begin{align}
	\vec{v} 
	= v^1 \hat{\vec{e}}_{(1)} + v^2\hat{\vec{e}}_{(2)}
	\ .
\end{align}
However, they find it super strange when we write
\begin{align}
	\vec{v} = 
	\begin{pmatrix}
		v^1\\
		v^2
	\end{pmatrix} \ .
\end{align}
The reason should be clear from this problem set. In a different basis, you have different components, $v'^i$ in \eqref{eq:primed}. Then we end up saying:
\begin{align}
	\vec{v} = 
	\begin{pmatrix}
		v^1\\
		v^2
	\end{pmatrix}
	=
	\begin{pmatrix}
		v'^1\\
		v'^2
	\end{pmatrix}
	 \ ,
\end{align}
which now looks completely absurd. Both columns represent the same vector $\vec{v}$. However, it looks like we're writing $v^1 = v'^1$ and $v^2 = v'^2$, which is completely wrong.

Of course, the point is that a vector is \emph{not} its components, but it is its components written with respect to an explicitly stated (or implicitly assumed) basis. There are different ways to indicate this, for example always explicitly writing the basis elements: \eqref{eq:primed}. 


\subsection*{Digging deeper}

One way to view rotations is as a change of basis from one orthonormal basis to another. This is what is called a \emph{passive transformation} because the vector itself doesn't change, it is just the reference basis. Define $R$ to be the transformation in the following way:
\begin{align}
	\vec{v} = v^i \hat{\vec{e}}_{(i)} = 
	R^i_{\phantom{i}j}v^j
	\;
	(R^{-1})^k_{\phantom{k}i}
	\hat{\vec{e}}_{(k)} \ .
\end{align}
You can see that the right-hand side is simply the same as the middle expression because
\begin{align}
	R^i_{\phantom{i}j} (R^{-1})^k_{\phantom{k}i} = \delta^k_j \ .
\end{align}
All we have done is transformed the components $v^i$ in one direction and then transformed the basis vectors $\hat{\vec{e}}_{(k)}$ in the opposite direction to compensate. The new basis vectors are
\begin{align}
	\hat{\vec{e}}'_{(i)} = (R^{-1})^k_{\phantom{k}i}
	\hat{\vec{e}}_{(k)} \ .
\end{align}
If we know the explicit form of $R$ (and thus of its inverse\footnote{Recall from lecture that if you have an invertible $N\times N$ matrix $R$, then you can readily write a system of $N^2$ equations whose solution gives the $N^2$ components of $R^{-1}$. If you have forgotten, then remind yourself of this fact by writing $R^{-1}R = 1$ in index notation and count how many equations there are.}), then we may explicitly find the new basis vectors with respect to the old ones.

Alternatively, if you have two sets of orthonormal basis vectors, then you can reconstruct the transformation $R^{-1}$ by taking dot products:
\begin{align}
	\hat{\vec{e}}'_{(i)}\cdot  
	\hat{\vec{e}}_{(\ell)}
	&= (R^{-1})^k_{\phantom{k}i}
	\hat{\vec{e}}_{(k)} 
	\cdot  
	\hat{\vec{e}}_{(\ell)}
	\\&=
	(R^{-1})^k_{\phantom{k}i}
	\delta_{k\ell}
	\\&=
	(R^{-1})^\ell_{\phantom{k}i}
	\ .
\end{align}
Compare this to your expressions in \eqref{eq:basisdots}. At this point, you should also be able to write the rotation matrix $R$ that acts on the components of the vectors, $v'^i = R^i_{\phantom{i}j}v^j$. 





% remark: vector isn't its components, can't say v=column.
\end{document}
