\documentclass[12pt]{article}
%% arXiv paper template by Flip Tanedo
%% last updated: Dec 2016



%%%%%%%%%%%%%%%%%%%%%%%%%%%%%
%%%  THE USUAL PACKAGES  %%%%
%%%%%%%%%%%%%%%%%%%%%%%%%%%%%

\usepackage{amsmath}
\usepackage{amssymb}
\usepackage{amsfonts}
\usepackage{graphicx}
\usepackage{xcolor}
\usepackage{nopageno}
\usepackage{enumerate}
\usepackage{parskip}
\usepackage{framed}

\usepackage{sectsty}
\sectionfont{\Large}
% \subsectionfont{\large}
% \renewcommand{\thesection}{}
% \renewcommand{\thesubsection}{\arabic{subsection}}

%%%%%%%%%%%%%%%%%%%%%%%%%%%%%%%%%%%%%%%%%%%%%%%
%%%  PAGE FORMATTING and (RE)NEW COMMANDS  %%%%
%%%%%%%%%%%%%%%%%%%%%%%%%%%%%%%%%%%%%%%%%%%%%%%

\usepackage[margin=2cm]{geometry}   % reasonable margins

\graphicspath{{figures/}}	        % set directory for figures

% for capitalized things
\newcommand\acro[1]{{\small {#1}}}

\numberwithin{equation}{section}    % set equation numbering
\renewcommand{\tilde}{\widetilde}   % tilde over characters
\renewcommand{\vec}[1]{\mathbf{#1}} % vectors are boldface

\newcommand{\dbar}{d\mkern-6mu\mathchar'26}    % for d/2pi
\newcommand{\ket}[1]{\left|#1\right\rangle}    % <#1|
\newcommand{\bra}[1]{\left\langle#1\right|}    % |#1>
\newcommand{\Xmark}{\text{\sffamily X}}        % cross out

\let\olditemize\itemize
\renewcommand{\itemize}{
  \olditemize
  \setlength{\itemsep}{1pt}
  \setlength{\parskip}{0pt}
  \setlength{\parsep}{0pt}
}


% Commands for temporary comments
\newcommand{\comment}[2]{\textcolor{red}{[\textbf{#1} #2]}}
\newcommand{\flip}[1]{{\color{red} [\textbf{Flip}: {#1}]}}
\newcommand{\email}[1]{\texttt{\href{mailto:#1}{#1}}}

\newenvironment{institutions}[1][2em]{\begin{list}{}{\setlength\leftmargin{#1}\setlength\rightmargin{#1}}\item[]}{\end{list}}


\usepackage{fancyhdr}		% to put preprint number



%%%%%%%%%%%%%%%%%%%
%%%  HYPERREF  %%%%
%%%%%%%%%%%%%%%%%%%

%% This package has to be at the end; can lead to conflicts
\usepackage{microtype}
\usepackage[
	colorlinks=true,
	citecolor=black,
	linkcolor=black,
	urlcolor=green!50!black,
	hypertexnames=false]{hyperref}





\begin{document}


\begin{center}

    {\Large \textsc{Short HW 5}:
    \textbf{Commutators, Degenerate Eigenvalues}}
    
\end{center}

\vskip .4cm

\noindent
\begin{tabular*}{\textwidth}{rl}
	\textsc{Course:}& Physics 017, \emph{Linear Algebra for Physics} (S22)
	\\
	\textsc{Instructor:}& Prof. Flip Tanedo (\email{flip.tanedo@ucr.edu})
	\\
	\textsc{Due by:}& \textbf{Thursday}, April 28 
\end{tabular*}

\noindent
Note that this short assignment is due by class on Thursday. You have only \emph{two days} to do it. This should be quick, I recommend doing it right after class on Tuesday.



The \textbf{commutator} of two matrices $A$ and $B$ is 
\begin{align}
	\left[A,B\right] \equiv AB - BA \ .
\end{align}
When $[A,B]=0$, we say that the two matrices commute. 

\section{Degenerate Eigenvectors}

Suppose a matrix has degenerate eigenvalues, $\lambda_i=\lambda_j$ for $i\neq j$. This leads to an ambiguity for the choice of eigenvectors. To see, this consider the symmetric matrix $A$,
\begin{align}
	A = \frac{1}{2}
	\begin{pmatrix}
		4 & 0 & 0\\
		0 & 3 & 1\\
		0 & 1 & 3
	\end{pmatrix}
	= 
	R_x 
	\begin{pmatrix}
		2 & 0 & 0 \\
		0 & 1 & 0 \\
		0 & 0 & 2
	\end{pmatrix}
	R_x^T \ .
	\label{eq:A}
\end{align}
Here $R_x$ is the rotation by angle $\pi/4$ with respect to the $x$-axis,
\begin{align}
	R_x = \begin{pmatrix}
		1 & 0 & 0 \\
		0 & \frac{1}{\sqrt{2}} & \frac{1}{\sqrt{2}}\\
		0 & \frac{-1}{\sqrt{2}} & \frac{1}{\sqrt{2}}
	\end{pmatrix} \ .
\end{align}
Show that the two eigenvectors with eigenvalue $2$ are not unique. That is to say, there are an infinite number of ways of writing a pair of eigenvectors of $A$ that (i) have eigenvalue $3/2$, (2) are orthonormal to each other, and (3) are orthonormal to the eigenvector with eigenvalue $1$. \textsc{Hint: the product of two rotation is a rotation.} 



\section{Simultaneously Diagonalizable Matrices}

Suppose $A$ and $B$ are both symmetric matrices. 

\subsection{Commutation Relation}

Show that if $A$ and $B$ are diagonalized by the same rotation, then the two matrices commute.
\begin{align}
	\left[A,B\right] = 0
\end{align}

\textsc{Hint:} diagonal matrices commute.

\textsc{Comment:} it is also true that if $A$ and $B$ commute, then they can be diagonalized by the same rotation. That is a little more tricky to prove, though.


\subsection{Breaking the Degeneracy}

Suppose $A$ is defined as in \eqref{eq:A}, and $B$ is define to be
\begin{align}
	B = %\frac{1}{2}
	\begin{pmatrix}
		1 & 0 & 0\\
		0 & 2 & 1\\
		0 & 1 & 2
	\end{pmatrix}
	= 
	R_x 
	\begin{pmatrix}
		1 & 0 & 0 \\
		0 & 1 & 0 \\
		0 & 0 & 3
	\end{pmatrix}
	R_x^T \ .
	\label{eq:B}
\end{align}
%
Observe that $B$, like $A$, has a degeneracy: two eigenvectors have the same eigenvalue, $\lambda=1$. Thus these two eigenvectors can be rotated into one another and the result are equally valid eigenvectors.

$A$ and $B$ can simultaneously be diagonalized. Let $\lambda_{A,i}$ be the $i^\text{th}$ eigenvalue of $A$, and similarly with $\lambda_{B,i}$ for $B$. We can label the eigenvectors in ket notation:
\begin{align}
	|\lambda_{A,i}, \lambda_{B,i} \rangle
\end{align}
is the eigenvector of both $A$ and $B$ that satisfies
\begin{align}
	A|\lambda_{A,i}, \lambda_{B,i} \rangle
	&= \lambda_{A,i}|\lambda_{A,i}, \lambda_{B,i} \rangle
	&
	B|\lambda_{A,i}, \lambda_{B,i} \rangle
	&=
	\lambda_{B,i}
	|\lambda_{A,i}, \lambda_{B,i} \rangle \ .
\end{align}
Write out the eigenvectors $|\lambda_{A,i}, \lambda_{B,i} \rangle$ that are the simultaneous eigenvectors of $A$ and $B$. Explain why there is no more degeneracy in how these eigenvectors are defined, even though both $A$ and $B$ had degeneracies in how to define the eigenvectors. 





% remark: vector isn't its components, can't say v=column.
\end{document}
