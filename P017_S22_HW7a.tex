\documentclass[12pt]{article}
%% arXiv paper template by Flip Tanedo
%% last updated: Dec 2016



%%%%%%%%%%%%%%%%%%%%%%%%%%%%%
%%%  THE USUAL PACKAGES  %%%%
%%%%%%%%%%%%%%%%%%%%%%%%%%%%%

\usepackage{amsmath}
\usepackage{amssymb}
\usepackage{amsfonts}
\usepackage{graphicx}
\usepackage{xcolor}
\usepackage{nopageno}
\usepackage{enumerate}
\usepackage{parskip}
\usepackage{framed}
\usepackage{bbm}

\usepackage{sectsty}
\sectionfont{\Large}
% \subsectionfont{\large}
% \renewcommand{\thesection}{}
% \renewcommand{\thesubsection}{\arabic{subsection}}

%%%%%%%%%%%%%%%%%%%%%%%%%%%%%%%%%%%%%%%%%%%%%%%
%%%  PAGE FORMATTING and (RE)NEW COMMANDS  %%%%
%%%%%%%%%%%%%%%%%%%%%%%%%%%%%%%%%%%%%%%%%%%%%%%

\usepackage[margin=2cm]{geometry}   % reasonable margins

\graphicspath{{figures/}}	        % set directory for figures

% for capitalized things
\newcommand\acro[1]{{\small {#1}}}

\numberwithin{equation}{section}    % set equation numbering
\renewcommand{\tilde}{\widetilde}   % tilde over characters
\renewcommand{\vec}[1]{\mathbf{#1}} % vectors are boldface

\newcommand{\dbar}{d\mkern-6mu\mathchar'26}    % for d/2pi
\newcommand{\ket}[1]{\left|#1\right\rangle}    % <#1|
\newcommand{\bra}[1]{\left\langle#1\right|}    % |#1>
\newcommand{\Xmark}{\text{\sffamily X}}        % cross out

\let\olditemize\itemize
\renewcommand{\itemize}{
  \olditemize
  \setlength{\itemsep}{1pt}
  \setlength{\parskip}{0pt}
  \setlength{\parsep}{0pt}
}


% Commands for temporary comments
\newcommand{\comment}[2]{\textcolor{red}{[\textbf{#1} #2]}}
\newcommand{\flip}[1]{{\color{red} [\textbf{Flip}: {#1}]}}
\newcommand{\email}[1]{\texttt{\href{mailto:#1}{#1}}}

\newenvironment{institutions}[1][2em]{\begin{list}{}{\setlength\leftmargin{#1}\setlength\rightmargin{#1}}\item[]}{\end{list}}


\usepackage{fancyhdr}		% to put preprint number



%%%%%%%%%%%%%%%%%%%
%%%  HYPERREF  %%%%
%%%%%%%%%%%%%%%%%%%

%% This package has to be at the end; can lead to conflicts
\usepackage{microtype}
\usepackage[
	colorlinks=true,
	citecolor=black,
	linkcolor=black,
	urlcolor=green!50!black,
	hypertexnames=false]{hyperref}





\begin{document}


\begin{center}

    {\Large \textsc{Short HW 7}:
    \textbf{Fourier Space}}
    
\end{center}

\vskip .4cm

\noindent
\begin{tabular*}{\textwidth}{rl}
	\textsc{Course:}& Physics 017, \emph{Linear Algebra for Physics} (S22)
	\\
	\textsc{Instructor:}& Prof. Flip Tanedo (\email{flip.tanedo@ucr.edu})
	\\
	\textsc{Due by:}& \textbf{Thursday}, May 12
\end{tabular*}

\noindent
Note that this short assignment is due by class on Thursday. You have only \emph{two days} to do it. This should be quick, I recommend doing it right after class on Tuesday.

Consider the space of functions $f(x)$ on interval from $x=0$ to $x=L$ with Dirichlet boundary conditions,
\begin{align}
 	f(0)&=0 & f(L)&= 0 \ .
 \end{align}
The \textbf{Fourier series} is a way of writing these functions in terms of a basis of sines or cosines. The choice of sines or cosines (or both) depends on the boundary conditions. For the above boundary conditions, the following basis of functions/kets/vectors is appropriate: 
\begin{align}
	|n\rangle  &=  C_n \sin (k_n x)  & n = 1, 2, 3, \cdots\ .
\end{align}
We sometimes call these basis vectors \emph{modes} of the Fourier series.
We are writing $|n\rangle$ instead of $\vec{e}_{(n)}$ or $|e_{n}\rangle$ for convenience; it's just a basis function. The inner product on this space between two functions $f(x)$ and $g(x)$ is
\begin{align}
	\langle f,g\rangle = \int_0^1 dx\, f(x) g(x) \ .
\end{align} 



\section{The basis vectors}

\begin{framed}
The following integrals may be handy:
 \begin{align}
 	\int_{0}^\pi dx\; \sin^2(\pi x) &= \frac{\pi}{2} 
 	&
	\int_{0}^\pi dx\; \sin(n \pi x) \sin(m \pi x) &= 0
 	\ ,
 \end{align}
 where in the second relation we assume $m\neq n$. {You can think about how to prove these relations. One clever way is to use $\cos^2\theta + \sin^2\theta = 1$ and make and argument based on periodicity.}
\end{framed}

\subsection{Finding the frequencies}

The $k_n$ is related to the angular frequency or momentum of each mode. In order to satisfy the boundary condition $f(L) = 0$, there are restrictions on the possible values of $k_n$: the frequencies must take on \emph{discrete} values such that $|n\rangle$ has a \emph{node} (it is zero) at $x=L$. In other words, \emph{the modes of a Fourier series have quantized frequencies}.

There are an infinite number of ways to pick $k_n$ to satisfy $f(L)=0$. The index $n$ enumerates the infinite allowed values of $k_n$. Derive the expression for the possible values of $k_n$. You may assume that $k_n>0$.\footnote{You should ask: why not negative values of $k_n$? This is a great question. Think what do negative frequency basis functions look like? Are those linearly independent from the positive frequency basis functions?}

\textsc{Answer}: The answer is $k_n = n\pi/L$ for $n=1,2,3,\cdots$. You should explain \emph{how} one arrives at this answer.


\subsection{Normalizing the Fourier basis}

Use the normalization condition $\langle n,n\rangle = 1$ to determine the basis prefactors $C_n$.

\textsc{Answer}: The answer is $C_n = \sqrt{2/L}$. You should explain \emph{how} one would arrive at this answer.


\subsection{What do the basis vectors look like?}

Sketch a graph of the first three basis vectors. Make sure your sketch only goes over the appropriate domain of $x$. 

 
\section{Fourier Series}

As is often the case, the key step to representing a function $f(x)$ in its Fourier series is to multiply by one:
\begin{align}
	\mathbbm{1} = \sum_{n=1}^\infty |n\rangle \langle n| \ ,
\end{align}
where we recall that the bra/row vector $\langle n|$ acts on a function $f(x)=|f\rangle$ as
\begin{align}
	\langle n | f\rangle = \langle n, f\rangle = \int_0^L dx\, C_n \sin(k_n x) f(x) \ .
\end{align}
The Fourier series representation of a function $f(x)$ in our function space is:
\begin{align}
	|f\rangle = \sum_{n=1}^\infty \langle n|f\rangle  |n\rangle  \ ,
\end{align}
where $\langle n|f\rangle$ are just numbers that are called the Fourier coefficients. To write it out more explicitly, this simply says:
\begin{align}
	f(x) &= \sum_{n=1}^\infty A_n \times C_n \sin(k_n x) & A_n = \langle n|f\rangle \ ,
	\end{align}
	where the $A_n$ are the Fourier coefficients.

Find the first three Fourier coefficients of the function $f(x) = x(x-L)$.\footnote{We have written $f(x) = x(x-1)$ rather than $x^2-x$ as a reminder that this is a function that satisfies the Dirichlet boundary conditions of our vector space.} You do not have to perform the integral, just write out what the $A_n$ are. The first one, for example, is
\begin{align}
	A_1 = \int_0^L x(x-1) C_1 \sin(k_1 x) \ ,
\end{align}
where you already know what the $C_1$ and $k_1$ are. You can always perform these integrals---for example, in \emph{Mathematica} if need be---but usually you can leave them implicit until you actually need a number. After all, these coefficients are just numbers with no functional dependence on $x$.


\section{So what?}

The Fourier series is convenient because it is a basis of \emph{eigenfunctions of the one-dimensional Laplacian}, $(d/dx)^2$. As we discussed in class, the Laplacian shows up all the times in physics because it is the rotationally symmetric [second] derivative that connects nearby points in space. What is the eigenvalues $\lambda_n$ of the $n^\text{th}$ Fourier basis vector, $|n\rangle$, with respect to $(d/dx)^2$?

\textsc{Comment:} The eigenvalues of a Hermitian operator like the Laplacian typically carries physical significance. For example, these may be the allowed harmonics on a guitar string, or the Kaluza--Klein modes of a particle in an extra dimension. 


\end{document}
