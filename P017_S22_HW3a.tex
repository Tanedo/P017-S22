\documentclass[12pt]{article}
%% arXiv paper template by Flip Tanedo
%% last updated: Dec 2016



%%%%%%%%%%%%%%%%%%%%%%%%%%%%%
%%%  THE USUAL PACKAGES  %%%%
%%%%%%%%%%%%%%%%%%%%%%%%%%%%%

\usepackage{amsmath}
\usepackage{amssymb}
\usepackage{amsfonts}
\usepackage{graphicx}
\usepackage{xcolor}
\usepackage{nopageno}
\usepackage{enumerate}
\usepackage{parskip}
\usepackage{framed}

\usepackage{sectsty}
\sectionfont{\Large}
% \subsectionfont{\large}
% \renewcommand{\thesection}{}
% \renewcommand{\thesubsection}{\arabic{subsection}}

%%%%%%%%%%%%%%%%%%%%%%%%%%%%%%%%%%%%%%%%%%%%%%%
%%%  PAGE FORMATTING and (RE)NEW COMMANDS  %%%%
%%%%%%%%%%%%%%%%%%%%%%%%%%%%%%%%%%%%%%%%%%%%%%%

\usepackage[margin=2cm]{geometry}   % reasonable margins

\graphicspath{{figures/}}	        % set directory for figures

% for capitalized things
\newcommand\acro[1]{{\small {#1}}}

\numberwithin{equation}{section}    % set equation numbering
\renewcommand{\tilde}{\widetilde}   % tilde over characters
\renewcommand{\vec}[1]{\mathbf{#1}} % vectors are boldface

\newcommand{\dbar}{d\mkern-6mu\mathchar'26}    % for d/2pi
\newcommand{\ket}[1]{\left|#1\right\rangle}    % <#1|
\newcommand{\bra}[1]{\left\langle#1\right|}    % |#1>
\newcommand{\Xmark}{\text{\sffamily X}}        % cross out

\let\olditemize\itemize
\renewcommand{\itemize}{
  \olditemize
  \setlength{\itemsep}{1pt}
  \setlength{\parskip}{0pt}
  \setlength{\parsep}{0pt}
}


% Commands for temporary comments
\newcommand{\comment}[2]{\textcolor{red}{[\textbf{#1} #2]}}
\newcommand{\flip}[1]{{\color{red} [\textbf{Flip}: {#1}]}}
\newcommand{\email}[1]{\texttt{\href{mailto:#1}{#1}}}

\newenvironment{institutions}[1][2em]{\begin{list}{}{\setlength\leftmargin{#1}\setlength\rightmargin{#1}}\item[]}{\end{list}}


\usepackage{fancyhdr}		% to put preprint number



%%%%%%%%%%%%%%%%%%%
%%%  HYPERREF  %%%%
%%%%%%%%%%%%%%%%%%%

%% This package has to be at the end; can lead to conflicts
\usepackage{microtype}
\usepackage[
	colorlinks=true,
	citecolor=black,
	linkcolor=black,
	urlcolor=green!50!black,
	hypertexnames=false]{hyperref}





\begin{document}


\begin{center}

    {\Large \textsc{Short HW 3}:
    \textbf{Gram--Schmidt}}
    
\end{center}

\vskip .4cm

\noindent
\begin{tabular*}{\textwidth}{rl}
	\textsc{Course:}& Physics 017, \emph{Linear Algebra for Physics} (S2022)
	\\
	\textsc{Instructor:}& Prof. Flip Tanedo (\email{flip.tanedo@ucr.edu})
	\\
	\textsc{Due by:}& \textbf{Thursday}, April 14 (yes, \emph{this} Thursday)
\end{tabular*}

\noindent
Note that this short assignment is due by class on Thursday. You have only \emph{two days} to do it. This should be quick, I recommend doing it right after class on Tuesday.


\section{Gram--Schmidt for a vectors in 3D Euclidean Space}

You are given three vectors that are \emph{linearly independent}\footnote{This means that you cannot write any vector as a linear combination of the other vectors. That is: each vector has at least some component that points in a `new' direction relative to the plane \emph{spanned} by the other vectors.}:
\begin{align}
\vec{v} &=
	\begin{pmatrix}
		1 \\ 0 \\ 3
	\end{pmatrix}
	&
\vec{w} &=
	\begin{pmatrix}
		1 \\ 1 \\ 2
	\end{pmatrix}
	&
\vec{z} &=
	\begin{pmatrix}
		-1 \\ -1 \\ \phantom{+}1
	\end{pmatrix}
\end{align}
Perform the Gram--Schmidt procedure to derive an orthonormal basis from these vectors. The first basis vector $\vec{e}_{(1)}$ should be parallel to $\vec{v}$. The second basis vector $\vec{e}_{(2)}$ should be on the $\vec{v}$--$\vec{w}$ plane. 





% remark: vector isn't its components, can't say v=column.
\end{document}
